\documentclass[12pt]{beamer}

\usetheme{Air}
\usepackage[czech]{babel}
\usepackage{thumbpdf}
\usepackage{wasysym}
\usepackage{ucs}
\usepackage[utf8]{inputenc}
\usepackage{verbatim}

\pdfinfo
{
  /Title       (Prostředí pro implementaci algoritmů Bayesovské filtrace)
%  /Creator     (TeX)
%  /Author      (Matěj Laitl)
}


\title{Prostředí pro implementaci algoritmů Bayesovské filtrace}
%\subtitle{TODO}
\author{Matěj Laitl\\ vedoucí práce: Ing. Václav Šmídl, Ph.D.; ÚTIA}
\date{\today}

\begin{document}

\frame{\titlepage}

% \section*{}
% \begin{frame}
%   \frametitle{Obsah}
%   \tableofcontents[section=1,hidesubsections]
% \end{frame}

\AtBeginSection[]
{
  \frame<handout:0>
  {
    \frametitle{Přehled}
    \tableofcontents[currentsection,hideallsubsections]
  }
}

\AtBeginSubsection[]
{
  \frame<handout:0>
  {
    \frametitle{Přehled}
    \tableofcontents[sectionstyle=show/hide,subsectionstyle=show/shaded/hide]
  }
}

\newcommand<>{\highlighton}[1]{%
  \alt#2{\structure{#1}}{{#1}}
}

\newcommand{\icon}[1]{\pgfimage[height=1em]{#1}}



%%%%%%%%%%%%%%%%%%%%%%%%%%%%%%%%%%%%%%%%%
%%%%%%%%%% Content starts here %%%%%%%%%%
%%%%%%%%%%%%%%%%%%%%%%%%%%%%%%%%%%%%%%%%%



\section{Úvod - zadání}

\begin{frame}
	\frametitle{Zadání}

	\begin{block}{Úkoly práce}
		\begin{itemize}
		\item seznámit se s metodou rekurzivní Bayesovské identifikace
		\item softwarová analýza - vhodnost přístupů a programových prostředí
		\item implementace několika metod Bayesovské filtrace
		\item prezentace implementace jako open-source projektu, zařazení do archivů sw
		\end{itemize}
	\end{block}
\end{frame}

\section{Rekurzivní Bayesovská identifikace}

\begin{frame}
	\frametitle{Formulace problému 1/2}
	\framesubtitle{Dynamický systém}

	\begin{block}{model procesu}
		\(x_t = f_t(x_{t-1}, v_{t-1})\) \hspace{3cm} \(x_t\) .. stavový vektor

		\begin{itemize}
		\item \(v_t\) náhodný šum, iid
		\item → definuje pdf\footnote{hustota pravděpodobnosti} \(p(x_t|x_{t-1})\)
		\end{itemize}
	\end{block}

	\begin{block}{model pozorování}
		\(y_t = h_t(x_t, w_t)\) \hspace{3cm} \(y_t\) .. vektor pozorování

		\begin{itemize}
		\item \(w_t\) náhodný šum, iid
		\item → definuje pdf \(p(y_t|x_t)\)
		\end{itemize}
	\end{block}
\end{frame}

\begin{frame}
	\frametitle{Formulace problému 2/2}
	\framesubtitle{Cíle}

	{\large Úkol:
	\begin{itemize}
		\item odhadnout \(x_t\) z posloupnosti meření \(y_{1:t}\)
		\item přesněji: spočítat pdf \(p(x_t | y_{1:t})\)
	\end{itemize}
	}

	\begin{block}{Máme}
		\begin{itemize}
			\item \(p(x_t|x_{t-1})\) \hspace{1cm} + \(p(x_0)\)
			\item \(p(y_t|x_t)\)
		\end{itemize}
	\end{block}

	\begin{example}
		\begin{itemize}
			\item robotika (Apollo 11)..
			\item počasí, síření radioaktivního spadu..
		\end{itemize}
	\end{example}
\end{frame}

\begin{frame}
	\frametitle{Matematický aparát}
	\begin{block}{Řetězové pravidlo pro pdf}
		\begin{align*}
			\hspace{-1cm}
			P(AB|C) &= P(A|BC) P(B|C)
		\end{align*}
	\end{block}

	\begin{block}{Bayesova věta}
		\begin{align*}
			P(A|BC) = \frac{P(B|AC)P(A|C)}{P(B|C)}
		\end{align*}
	\end{block}

	\begin{block}{Marginalizace pdf}
		\begin{align*}
			p(x, y) = \int_{-\infty}^{\infty} p(x, y, z) \; \mathrm{d} z
		\end{align*}
	\end{block}
\end{frame}

\begin{frame}
	\frametitle{Odvození řešení 1/2}
	\framesubtitle{a priori pdf}

	Máme \(p(x_{t-1} | y_{1:t-1})\) (rekurze), v prvním kroku odvodíme \(p(x_t | y_{1:t-1})\) 
	(bez znalosti nového měření \(y_t\)). (řetěz. pravidlo, marginalizace)

	\begin{align*}
		p(x_t | y_{1:t-1}) &= \int_{-\infty}^{\infty} p(x_t, x_{t-1} | y_{1:t-1}) \; \mathrm{d} x_{t-1} \\
		%
		p(x_t | y_{1:t-1}) &= \int_{-\infty}^{\infty} p(x_t | x_{t-1}, y_{1:t-1}) p(x_{t-1} | y_{1:t-1}) \; \mathrm{d} x_{t-1} \\
		%
		p(x_t | y_{1:t-1}) &= \int_{-\infty}^{\infty} p(x_t | x_{t-1}) p(x_{t-1} | y_{1:t-1}) \; \mathrm{d} x_{t-1}
	\end{align*}
\end{frame}

\begin{frame}
	\frametitle{Odvození řešení 2/2}
	\framesubtitle{a posteriori pdf}

	Přišlo pozorování \(y_t\), zakomponujme ho. (Bayes věta)

	\begin{align*}
		p(x_t | y_{1:t}) &= \frac{p(y_t | x_t, y_{1:t-1}) p(x_t | y_{1:t-1})}{p(y_t | y_{1:t-1})} \\
		%
		p(x_t | y_{1:t}) &= \frac{p(y_t | x_t) p(x_t | y_{1:t-1})}{p(y_t | y_{1:t-1})}
	\end{align*}

	Čitatel známý, jmenovatel je konstanta - buď netřeba nebo lze dopočítat marginalizací/z podmínky
	normality.
\end{frame}

\begin{frame}
	\frametitle{Problémy analytického řešení}

	{\large Odvozené analytické řešení nelze dost dobře počítat} (vícerozměrný integrál)

	\begin{block}{Zjednodušení problému}
		\begin{itemize}
			\item Kalmanův filtr: a priori i posteriori pdf jsou Gaussovky, šumy jsou Gaussovské
			centrované, lineární systém
			\item Síťové metody: pro diskrétní stavy
		\end{itemize}
	\end{block}

	\begin{block}{Aproximace}
		\begin{itemize}
			\item Particle filter\footnote{překlad ``částicový filtr'' se nepoužívá}: odhaduje aposteriori pdf empirickou pdf,
			nedeterministický (Monte Carlo - samplování z pdf)
			\item Marginalizovaný Particle Filter
		\end{itemize}
	\end{block}

\end{frame}

\section{Výsledky softwarové analýzy}

\begin{frame}
	\frametitle{Východiska}

	Cíl: vyvinout použitelnou knihovnu vhodnou pro vývoj nových přístupů Bayesovského
	odhadování/rozhodování

 	\begin{block}{Požadavky}
		\begin{itemize}
			\item snadné a rychlé použití (vysokoúrovňový jazyk)
			\item dostatečná rychlost
			\item nízké počáteční požadavky na uživatele
			\item přenositelnost, nezávislost na komerčních platformách
		\end{itemize}
 	\end{block}
\end{frame}

\begin{frame}
	\frametitle{Paradigmata, Programovací jazyky}

	\begin{block}{Vhodnost implementačních metodik}
		Objektově orientované programování nejlépe vystihuje matematickou strukturu teorie
	\end{block}

	\begin{block}{Implementační prostředí 1}
		\begin{itemize}
			\item C++
			\begin{itemize}
				\item + rozšířenost, rychlost, existující knihovny pro Bayes..
				\item - nízká produktivita, vysoké vstupní bariéry - nevhodné
			\end{itemize}
			\item Matlab
			\begin{itemize}
				\item + rozšířenost, vysoká produktivita
				\item - CoW, problematický objektový model, proprietární - nevhodné
			\end{itemize}
		\end{itemize}
	\end{block}
\end{frame}

\begin{frame}
	\frametitle{Programovací jazyky redux}

	\begin{block}{Implementační prostředí 2}
		\begin{itemize}
			\item Python
			\begin{itemize}
				\item + vysoká produktivita, existující mat. knihovny (NumPy), rozšíření, komunita
				\item - vysoká režie (interpretovaný jazyk) - sám o sobě nevhodný
			\end{itemize}
			\item Cython
			\begin{itemize}
				\item překládá python kód do C
				\item postupná optimalizace stávajícího python kódu
				\item naměřeno: 100x zrychlení pro numerickou integraci\footnote{dokazuje zejména pomalost pythonu}
				\item naměřeno: 13x zrychlení při paralelizaci na 16-jádrovém CPU (OpenMP)
			\end{itemize}
		\end{itemize}
	\end{block}
\end{frame}

\section{PyBayes}

\begin{frame}
	\frametitle{PyBayes}

	\begin{block}{PyBayes knihovna}
		\begin{itemize}
			\item open-source, otevřený vývoj na \url{http://github.com/strohel/PyBayes}
			\item python + cython (překlad pythonu do C: 2x -- 100x zrychlení)
		\end{itemize}
	\end{block}

	\begin{block}{Implementováno v PyBayes}
		\begin{itemize}
			\item základní i podmíněné pdf, produkt pdf, řetězové pravidlo
			\item Kalman Filter, Particle filter, Marginalizovaný PF
		\end{itemize}
	\end{block}
\end{frame}

\begin{frame}
  \vspace{2cm}
  {\huge Děkuji za pozornost}

  \vspace{3cm}
  \begin{flushright}
    Matěj Laitl

    \structure{\footnotesize{matej@laitl.cz}}
  \end{flushright}
\end{frame}

\end{document}
