\chapter{Software analysis}

\section{Requirements}

Ideal library for Bayesian filtering would posses following properties...:

\section{Programming paradigms}

Interpreted vs. Compiled

Object-oriented, procedural and Functional

pass-by reference vs. copy-on-write (Matlab)

\section{Survey of Existing Libraries for Bayesian estimation/decision making}

Brief survey of existing libraries ... not fullfilling all requirements..

The need to implement a new one :-)

\section{C++}

BDM

 - advantages of C (speed, C prevalence (many optimised libraries, BLAS, LAPACK.., OpenMP)

 - disadvantages of C in our ``situation'' (steep learning curve, coplexity because of low-levelness
   high initial barriers (need to have compiler, libraries...), inconveniently long edit/build/test
   process)

\section{Matlab}

BDM (partially?)

 - advantages (popularity, existing toolboxes, rapid development (high-level)

 - disadv: strict copy-on-write, problematic object model (not in original design), difficulties
           interfacing existing C (F) code

\section{Python}

NumPy.... parallelisation (approaches, improvements in Py 3.2) - GIL.. Py3k

\section{Cython}

general info etc... extension types, building, ease of interfacing C (and F) code, .pxd files,
NumPy support

[citations:\cite{BehBraSel:09,Sel:09,BehBraCitDalSelSmi:11}]

\subsection{Gradual Optimisation}

how can optimisaion be approached (gradually) and why this approach is superior

integrate\_python\_cython example (``100x'' speedup for a special (very simple) case)

\subsection{Parallelisation}

integrate\_python\_cython patched with OpenMP (13x speedup in 16-core system)

prange CEP

\subsection{Pure Python mode}

About it and why it should be used in a hypothetical bayesian python library

\subsection{Limitations}

2 types:

	not-supported code (few cases, but bad, ongoing work)

	not-optimised code (much more work needed, but not hard to fix in most cases)

		- exception handling (functions returning void etc)

		- limitations of pure python mode in regards to traditional .pyx files

\section{Choice}

python/cython was choosen ...
